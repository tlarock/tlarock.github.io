\documentclass[11pt]{article}

\newcommand{\yourname}{}
\newcommand{\yourcollaborators}{}

\def\comments{1}

\newcommand{\course}{\href{tlarock.github.io/teaching/cs3000/syllabus.html}{CS 3000: Algorithms \& Data}}
\newcommand{\instructor}{Tim LaRock}
\newcommand{\semester}{Summer 1 '20}

\newcommand{\hwnum}{5}
\newcommand{\hwdue}{Tuesday June 9th at 11:59pm via Gradescope}

%format and packages

\usepackage{graphicx}
%\usepackage{algorithm, algorithmic}
\usepackage{algpseudocode}
\usepackage{amsmath, amssymb, amsthm}
\usepackage{enumerate}
\usepackage{enumitem}
\usepackage{multirow}
\usepackage{framed}
\usepackage{verbatim}
\usepackage[margin=1.1in]{geometry}
\usepackage{microtype}
\usepackage{kpfonts}
\usepackage{palatino}
\DeclareMathAlphabet{\mathtt}{OT1}{cmtt}{m}{n}
\SetMathAlphabet{\mathtt}{bold}{OT1}{cmtt}{bx}{n}
\DeclareMathAlphabet{\mathsf}{OT1}{cmss}{m}{n}
\SetMathAlphabet{\mathsf}{bold}{OT1}{cmss}{bx}{n}
\renewcommand*\ttdefault{cmtt}
\renewcommand*\sfdefault{cmss}
\renewcommand{\baselinestretch}{1.05}
\usepackage[usenames,dvipsnames]{xcolor}
\definecolor{DarkGreen}{rgb}{0.15,0.5,0.15}
\definecolor{DarkRed}{rgb}{0.6,0.2,0.2}
\definecolor{DarkBlue}{rgb}{0.2,0.2,0.6}
\definecolor{DarkPurple}{rgb}{0.4,0.2,0.4}
\usepackage[pdftex]{hyperref}
\hypersetup{
	linktocpage=true,
	colorlinks=true,				% false: boxed links; true: colored links
	linkcolor=DarkBlue,		% color of internal links
	citecolor=DarkBlue,	% color of links to bibliography
	urlcolor=DarkBlue,		% color of external links
}

\usepackage{graphicx}
\usepackage{tikz}
\usetikzlibrary{positioning}
\definecolor{processblue}{cmyk}{0.96,0,0,0}
\usetikzlibrary{matrix,arrows}

\ifnum\comments=1
\newcommand{\mynote}[2]{\marginpar{{\color{#1} \tiny #2}}}
\newcommand{\mybignote}[2]{{\color{#1} $\langle \langle$ #2~$\rangle \rangle$}}
\else
\newcommand{\mynote}[2]{}
\newcommand{\mybignote}[2]{}
\fi
\newcommand{\jnote}[1]{\mynote{red}{Jon: {#1}}}
\newcommand{\bigjnote}[1]{\mybignote{red}{Jon: #1}}


\usepackage[boxruled,vlined,nofillcomment]{algorithm2e}
\SetKwProg{Fn}{Function}{\string:}{}
\SetKwFor{While}{While}{}{}
\SetKwFor{For}{For}{}{}
\SetKwIF{If}{ElseIf}{Else}{If}{:}{ElseIf}{Else}{:}
\SetKw{Return}{Return}


%enclosure macros
\newcommand{\paren}[1]{\ensuremath{\left( {#1} \right)}}
\newcommand{\bracket}[1]{\ensuremath{\left\{ {#1} \right\}}}
\renewcommand{\sb}[1]{\ensuremath{\left[ {#1} \right\]}}
\newcommand{\ab}[1]{\ensuremath{\left\langle {#1} \right\rangle}}

%probability macros
\newcommand{\ex}[2]{{\ifx&#1& \mathbb{E} \else \underset{#1}{\mathbb{E}} \fi \left[#2\right]}}
\newcommand{\pr}[2]{{\ifx&#1& \mathbb{P} \else \underset{#1}{\mathbb{P}} \fi \left[#2\right]}}
\newcommand{\var}[2]{{\ifx&#1& \mathrm{Var} \else \underset{#1}{\mathrm{Var}} \fi \left[#2\right]}}

\newcommand{\opt}{\textsc{opt}}

%useful CS macros
\newcommand{\poly}{\mathrm{poly}}
\newcommand{\polylog}{\mathrm{polylog}}
\newcommand{\zo}{\{0,1\}}
\newcommand{\pmo}{\{\pm1\}}
\newcommand{\getsr}{\gets_{\mbox{\tiny R}}}
\newcommand{\card}[1]{\left| #1 \right|}
\newcommand{\set}[1]{\left\{#1\right\}}
\newcommand{\negl}{\mathrm{negl}}
\newcommand{\eps}{\varepsilon}
\DeclareMathOperator*{\argmin}{arg\,min}
\DeclareMathOperator*{\argmax}{arg\,max}
\newcommand{\eqand}{\qquad \textrm{and} \qquad}
\newcommand{\ind}[1]{\mathbb{I}\{#1\}}
\newcommand{\sslash}{\ensuremath{\mathbin{/\mkern-3mu/}}}

%info theory macros
\newcommand{\SD}{\mathit{SD}}
\newcommand{\sd}[2]{\SD\left( #1 , #2 \right)}
\newcommand{\KL}{\mathit{KL}}
\newcommand{\kl}[2]{\KL\left(#1 \| #2 \right)}
\newcommand{\CS}{\ensuremath{\chi^2}}
\newcommand{\cs}[2]{\CS\left(#1 \| #2 \right)}
\newcommand{\MI}{\mathit{I}}
\newcommand{\mi}[2]{\MI\left(~#1~;~#2~\right)}

%mathbb
\newcommand{\N}{\mathbb{N}}
\newcommand{\R}{\mathbb{R}}
\newcommand{\Z}{\mathbb{Z}}
%mathcal
\newcommand{\cA}{\mathcal{A}}
\newcommand{\cB}{\mathcal{B}}
\newcommand{\cC}{\mathcal{C}}
\newcommand{\cD}{\mathcal{D}}
\newcommand{\cE}{\mathcal{E}}
\newcommand{\cF}{\mathcal{F}}
\newcommand{\cL}{\mathcal{L}}
\newcommand{\cM}{\mathcal{M}}
\newcommand{\cO}{\mathcal{O}}
\newcommand{\cP}{\mathcal{P}}
\newcommand{\cQ}{\mathcal{Q}}
\newcommand{\cR}{\mathcal{R}}
\newcommand{\cS}{\mathcal{S}}
\newcommand{\cU}{\mathcal{U}}
\newcommand{\cV}{\mathcal{V}}
\newcommand{\cW}{\mathcal{W}}
\newcommand{\cX}{\mathcal{X}}
\newcommand{\cY}{\mathcal{Y}}
\newcommand{\cZ}{\mathcal{Z}}

%theorem macros
\newtheorem{thm}{Theorem}
\newtheorem{lem}[thm]{Lemma}
\newtheorem{fact}[thm]{Fact}
\newtheorem{clm}[thm]{Claim}
\newtheorem{rem}[thm]{Remark}
\newtheorem{coro}[thm]{Corollary}
\newtheorem{prop}[thm]{Proposition}
\newtheorem{conj}[thm]{Conjecture}
\theoremstyle{definition}
\newtheorem{defn}[thm]{Definition}


\theoremstyle{theorem}
\newtheorem{prob}{Problem}
\newtheorem{sol}{Solution}

\definecolor{cit}{rgb}{0.05,0.2,0.45} 
\newcommand{\solution}{\medskip\noindent{\color{DarkBlue}\textbf{Solution:}}}

\begin{document}
	{\Large 
		\begin{center} \course\ --- \semester\ --- \instructor \end{center}}
	{\large
		\vspace{10pt}
		\noindent Homework~\hwnum \vspace{2pt}\\
		Due~\hwdue}
	
	\bigskip
	{\large
		\noindent Name: \yourname \vspace{2pt}\\ Collaborators: \yourcollaborators}
	
	\vspace{15pt}
	\begin{itemize}
		
		\item Make sure to put your name on the first page.  If you are using the \LaTeX~template we provided, then you can make sure it appears by filling in the \texttt{yourname} command.
		
		\item This assignment is due~\hwdue.  No late assignments will be accepted.  Make sure to submit something before the deadline.
		
		\item Solutions must be typeset in \LaTeX.  If you need to draw any diagrams, you may draw them by hand as long as they are embedded in the PDF.  I recommend using the source file for this assignment to get started.
		
		\item I encourage you to work with your classmates on the homework problems. \emph{If you do collaborate, you must write all solutions by yourself, in your own words.}  Do not submit anything you cannot explain.  Please list all your collaborators in your solution for each problem by filling in the \texttt{yourcollaborators} command.
		
		\item Finding solutions to homework problems on the web, or by asking students not enrolled in the class is strictly forbidden.
		
	\end{itemize}


	\newpage
		\begin{prob} Betweenness Centrality Practice\end{prob}

In this question, you will compute the betweenness centrality of the nodes $u$ and $c$ in the graph below.

\vspace{15pt}
\begin{figure}[h!]
	\begin{center}
		\begin{tikzpicture}[node distance = 1.5 cm and 2.5cm,on grid, thick,state/.style ={circle,top color =white,bottom color = gray!20,draw,black,text=black,minimum width =.3 cm}]
		\node[state] (0) at (-4, 3.5) {b};
		\node[state] (1) at (-5, 2.5) {a};
		\node[state] (2) at (-3, 2.5) {c};
		\node[state] (3) at (-4, 1.5) {d};
		\node[state] (4) at (-1, 2.5) {u};
		\node[state] (5) at (1, 2.5) {e};
		\node[state] (6) at (2, 3.5) {f};
		\node[state] (8) at (2, 1.5) {g};
		\tikzset{every node/.style={fill=white}} 
		\draw (1) -- (2);
		\draw (0) -- (2);
		\draw (2) -- (3);
		\draw (3) -- (1);
		\draw (0) -- (3);
		\draw (5) -- (8);
		\draw (6) -- (5);
		\draw (5) -- (4);
		\draw (4) -- (2);
		\end{tikzpicture}
		
	\end{center}
\end{figure}

\vspace{15pt}
\begin{enumerate}[label=(\alph*)]
	\item There are some pairs of nodes in this graph for which there is no possibility that the shortest path between them could include $u$. For example, the shortest path between nodes $a$ and $d$ is simply the edge between them. List all such pairs of nodes.
	
	\solution
	\begin{itemize}
		\item (a,d)
		\item (..)
	\end{itemize}
	
	\item List all of the shortest paths between the node $a$ and any node not included in the above list. If the shortest path contains $u$, use bold font for it, as in (v, w, \textbf{u}, x, y).
	
	\solution
	\begin{itemize}
		\item (v, y)
		\begin{itemize}
			\item (v, w, \textbf{u}, x, y)
		\end{itemize}
		
		
	\end{itemize}
	
	\item Compute the betweenness centrality of $u$ (reporting the number is enough, you do not need to show more work).
	
	\solution


	\item Compute the betweenness centrality of $c$ (again, you do not need to show work if you do not want to, you can just compute and report the value).
	
	\solution
	
\end{enumerate}
	
	\vfill\newpage
	
	\begin{prob} MST Practice \end{prob}
	
	Compute an MST in the following graph.  You do not need to justify your answer.  {\bf Hint:} Type your solution neatly by copying the \LaTeX~for the MST and removing some edges.
	
	\vspace{15pt}
	\begin{figure}[h!]
		\begin{center}
			\begin{tikzpicture}[node distance = 1.5 cm and 2.5cm,on grid, thick,state/.style ={circle,top color =white,bottom color = gray!20,draw,black,text=black,minimum width =.3 cm}]
			\node[state] (a) {1};
			\node[state] (b) [right =of a] {2};
			\node[state] (c) [below =of a, xshift=-40pt, yshift=-15pt] {3};
			\node[state] (d) [below =of b, xshift=+40pt, yshift=-15pt] {4};
			\node[state] (e) [below =of c, xshift=+40pt, yshift=-15pt] {5};
			\node[state] (f) [right =of e] {6};
			\tikzset{every node/.style={fill=white}} 
			\path 	(a) edge node {10} 	(b)
			(a) edge node {7} 		(c)
			(a) edge node {1}	 	(d)
			(a) edge node {6}		(f);
			\path 	(b) edge node {3}		(d)
			(b) edge node {2}	 	(f);
			\path 	(c) edge node {9}		(e)
			(c) edge node {5}	 	(f);
			\path 	(d) edge node {4} 		(f);
			\path 	(e) edge node {8} 		(f);
			\end{tikzpicture}
		\end{center}
	\end{figure}
	
	\vspace{15pt}
	\solution
	
	\vfill\newpage
	\begin{prob}Flow Practice\end{prob}
	
	Consider the following flow network $G = (V,E,\{c(e)\},s,t)$.
	
	\begin{figure}[h!]
		\begin{center}
			\begin{tikzpicture}[> = stealth, node distance = 2.0 cm and 3.0 cm, on grid, thick, state/.style = {circle, top color =white, bottom color = gray!20, draw, black, text = black, minimum width = .3 cm}]
			\node[state] (s) {s};
			\node[state] (a) [above right =of s, yshift=15pt] {1};
			\node[state] (b) [right =of s] {2};
			\node[state] (c) [right =of b] {3};
			\node[state] (d) [right =of c] {4};
			\node[state] (e) [above =of d, yshift=15pt] {5};
			\node[state] (f) [right =of d] {t};
			\tikzset{every node/.style={fill=white}} 
			\path[->] (s) 	edge node {10} (a)
			edge node {8} (b);
			\path[->] (a) 	edge node {4} (e)
			edge node {7} (b);
			\path[->] (b) 	edge node {6} (c)
			edge node {8} (e);
			\path[->] (c) 	edge node {4} (d);
			\path[->] (d) 	edge node {12} (f);
			\path[->] (e) 	edge node {9} (c)
			edge node {7} (d)
			edge node {2} (f);			
			\end{tikzpicture}
		\end{center}
	\end{figure}
	
	\begin{enumerate}[label=(\alph*)]
		
		\item Compute a maximum s-t flow in this network and its value.  State the sequence of augmenting paths used to obtain the flow.  {\bf Hint:} Type your solution neatly by copying the \LaTeX~for the flow network and labeling edges with their flow values $f(e)$ (preferably even with $f(e) / c(e)$ to help you keep track of the capacity constraints.)
		
		
		\solution
		
		\item Compute a minimum s-t cut in this network and its capacity.
		
		\solution
		
	\end{enumerate}
	
\end{document}