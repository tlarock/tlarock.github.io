\documentclass[11pt]{article}

\newcommand{\yourname}{}
\newcommand{\yourcollaborators}{}

\def\comments{1}

\newcommand{\course}{\href{tlarock.github.io/teaching/cs3000/syllabus.html}{CS 3000: Algorithms \& Data}}
\newcommand{\instructor}{Tim LaRock}
\newcommand{\semester}{Summer 1 '20}

\newcommand{\hwnum}{5}
\newcommand{\hwdue}{Monday June 1st at 11:59pm Boston time via Gradescope}

%format and packages

\usepackage{graphicx}
%\usepackage{algorithm, algorithmic}
\usepackage{algpseudocode}
\usepackage{amsmath, amssymb, amsthm}
\usepackage{enumerate}
\usepackage{enumitem}
\usepackage{multirow}
\usepackage{framed}
\usepackage{verbatim}
\usepackage[margin=1.1in]{geometry}
\usepackage{microtype}
\usepackage{kpfonts}
\usepackage{palatino}
	\DeclareMathAlphabet{\mathtt}{OT1}{cmtt}{m}{n}
	\SetMathAlphabet{\mathtt}{bold}{OT1}{cmtt}{bx}{n}
	\DeclareMathAlphabet{\mathsf}{OT1}{cmss}{m}{n}
	\SetMathAlphabet{\mathsf}{bold}{OT1}{cmss}{bx}{n}
	\renewcommand*\ttdefault{cmtt}
	\renewcommand*\sfdefault{cmss}
	\renewcommand{\baselinestretch}{1.05}
\usepackage[usenames,dvipsnames]{xcolor}
\definecolor{DarkGreen}{rgb}{0.15,0.5,0.15}
\definecolor{DarkRed}{rgb}{0.6,0.2,0.2}
\definecolor{DarkBlue}{rgb}{0.2,0.2,0.6}
\definecolor{DarkPurple}{rgb}{0.4,0.2,0.4}
\usepackage[pdftex]{hyperref}
\hypersetup{
	linktocpage=true,
	colorlinks=true,				% false: boxed links; true: colored links
	linkcolor=DarkBlue,		% color of internal links
	citecolor=DarkBlue,	% color of links to bibliography
	urlcolor=DarkBlue,		% color of external links
}

\usepackage{graphicx}
\usepackage{tikz}
\usetikzlibrary{positioning}
\definecolor{processblue}{cmyk}{0.96,0,0,0}
\usetikzlibrary{matrix,arrows}

\ifnum\comments=1
\newcommand{\mynote}[2]{\marginpar{{\color{#1} \tiny #2}}}
\newcommand{\mybignote}[2]{{\color{#1} $\langle \langle$ #2~$\rangle \rangle$}}
\else
\newcommand{\mynote}[2]{}
\newcommand{\mybignote}[2]{}
\fi
\newcommand{\jnote}[1]{\mynote{red}{Jon: {#1}}}
\newcommand{\bigjnote}[1]{\mybignote{red}{Jon: #1}}


\usepackage[boxruled,vlined,nofillcomment]{algorithm2e}
	\SetKwProg{Fn}{Function}{\string:}{}
	\SetKwFor{While}{While}{}{}
	\SetKwFor{For}{For}{}{}
	\SetKwIF{If}{ElseIf}{Else}{If}{:}{ElseIf}{Else}{:}
	\SetKw{Return}{Return}
	

%enclosure macros
\newcommand{\paren}[1]{\ensuremath{\left( {#1} \right)}}
\newcommand{\bracket}[1]{\ensuremath{\left\{ {#1} \right\}}}
\renewcommand{\sb}[1]{\ensuremath{\left[ {#1} \right\]}}
\newcommand{\ab}[1]{\ensuremath{\left\langle {#1} \right\rangle}}

%probability macros
\newcommand{\ex}[2]{{\ifx&#1& \mathbb{E} \else \underset{#1}{\mathbb{E}} \fi \left[#2\right]}}
\newcommand{\pr}[2]{{\ifx&#1& \mathbb{P} \else \underset{#1}{\mathbb{P}} \fi \left[#2\right]}}
\newcommand{\var}[2]{{\ifx&#1& \mathrm{Var} \else \underset{#1}{\mathrm{Var}} \fi \left[#2\right]}}

\newcommand{\opt}{\textsc{opt}}

%useful CS macros
\newcommand{\poly}{\mathrm{poly}}
\newcommand{\polylog}{\mathrm{polylog}}
\newcommand{\zo}{\{0,1\}}
\newcommand{\pmo}{\{\pm1\}}
\newcommand{\getsr}{\gets_{\mbox{\tiny R}}}
\newcommand{\card}[1]{\left| #1 \right|}
\newcommand{\set}[1]{\left\{#1\right\}}
\newcommand{\negl}{\mathrm{negl}}
\newcommand{\eps}{\varepsilon}
\DeclareMathOperator*{\argmin}{arg\,min}
\DeclareMathOperator*{\argmax}{arg\,max}
\newcommand{\eqand}{\qquad \textrm{and} \qquad}
\newcommand{\ind}[1]{\mathbb{I}\{#1\}}
\newcommand{\sslash}{\ensuremath{\mathbin{/\mkern-3mu/}}}

%info theory macros
\newcommand{\SD}{\mathit{SD}}
\newcommand{\sd}[2]{\SD\left( #1 , #2 \right)}
\newcommand{\KL}{\mathit{KL}}
\newcommand{\kl}[2]{\KL\left(#1 \| #2 \right)}
\newcommand{\CS}{\ensuremath{\chi^2}}
\newcommand{\cs}[2]{\CS\left(#1 \| #2 \right)}
\newcommand{\MI}{\mathit{I}}
\newcommand{\mi}[2]{\MI\left(~#1~;~#2~\right)}

%mathbb
\newcommand{\N}{\mathbb{N}}
\newcommand{\R}{\mathbb{R}}
\newcommand{\Z}{\mathbb{Z}}
%mathcal
\newcommand{\cA}{\mathcal{A}}
\newcommand{\cB}{\mathcal{B}}
\newcommand{\cC}{\mathcal{C}}
\newcommand{\cD}{\mathcal{D}}
\newcommand{\cE}{\mathcal{E}}
\newcommand{\cF}{\mathcal{F}}
\newcommand{\cL}{\mathcal{L}}
\newcommand{\cM}{\mathcal{M}}
\newcommand{\cO}{\mathcal{O}}
\newcommand{\cP}{\mathcal{P}}
\newcommand{\cQ}{\mathcal{Q}}
\newcommand{\cR}{\mathcal{R}}
\newcommand{\cS}{\mathcal{S}}
\newcommand{\cU}{\mathcal{U}}
\newcommand{\cV}{\mathcal{V}}
\newcommand{\cW}{\mathcal{W}}
\newcommand{\cX}{\mathcal{X}}
\newcommand{\cY}{\mathcal{Y}}
\newcommand{\cZ}{\mathcal{Z}}

%theorem macros
\newtheorem{thm}{Theorem}
\newtheorem{lem}[thm]{Lemma}
\newtheorem{fact}[thm]{Fact}
\newtheorem{clm}[thm]{Claim}
\newtheorem{rem}[thm]{Remark}
\newtheorem{coro}[thm]{Corollary}
\newtheorem{prop}[thm]{Proposition}
\newtheorem{conj}[thm]{Conjecture}
	\theoremstyle{definition}
\newtheorem{defn}[thm]{Definition}


\theoremstyle{theorem}
\newtheorem{prob}{Problem}
\newtheorem{sol}{Solution}

\definecolor{cit}{rgb}{0.05,0.2,0.45} 
\newcommand{\solution}{\medskip\noindent{\color{DarkBlue}\textbf{Solution:}}}

\begin{document}
{\Large 
\begin{center} \course\ --- \semester\ --- \instructor \end{center}}
{\large
\vspace{10pt}
\noindent Homework~\hwnum \vspace{2pt}\\
Due~\hwdue}

\bigskip
{\large
\noindent Name: \yourname \vspace{2pt}\\ Collaborators: \yourcollaborators}

\vspace{15pt}
\begin{itemize}
	
	\item Make sure to put your name on the first page.  If you are using the \LaTeX~template we provided, then you can make sure it appears by filling in the \texttt{yourname} command.
	
	\item This assignment is due~\hwdue.  Make sure to submit something before the deadline.
	
	\item Solutions must be typeset in \LaTeX.  If you need to draw any diagrams, you may draw them by hand as long as they are embedded in the PDF.  I recommend using the source file for this assignment to get started.
	
	\item I encourage you to work with your classmates on the homework problems. \emph{If you do collaborate, you must write all solutions by yourself, in your own words.}  Do not submit anything you cannot explain.  Please list all your collaborators in your solution for each problem by filling in the \texttt{yourcollaborators} command.
	
	\item Finding solutions to homework problems on the web, or by asking students not enrolled in the class, is strictly forbidden.
	
\end{itemize}

\newpage
\begin{prob}Graph Representations and Exploration\end{prob}

This problem tests your understanding of basic graph algorithms and concepts.

\begin{enumerate} [label=(\alph*)]
	\item Consider the following graph
\begin{figure}[h!]
	\begin{center}
		\begin{tikzpicture}[node distance = 1.0 cm and 2.0cm,on grid,
		thick,state/.style ={circle,top color =white,bottom color = processblue!20,draw,processblue,text=blue,minimum width =.3 cm}]
		\node[state] (a) {1};
		\node[state] (b) [yshift = 30pt, right =of a] {2};
		\node[state] (c) [yshift = -30pt, right =of b] {3};
		\node[state] (d) [below =of a, yshift=-14pt] {4};
		\node[state] (e) [yshift = 20pt, right =of d] {5};
		\node[state] (f) [yshift = -20pt, right =of e] {6};
		\node[state] (g) [below =of e, xshift=-27pt,yshift=-25pt] {7};
		\node[state] (h) [right =of g] {8};
		\path (a) edge [bend left = 0] (b) 
		edge [bend left = 0] (c) 
		edge (d);
		\path (b) edge [bend left = 0] (f);
		\path (c) edge [bend left = 0] (e);
		\path (d) edge [bend right = 50] (c) 
		edge [bend left = 0] (e) 
		edge (d);
		\path (e) edge [bend left = 0] (h);
		\path (g) edge [bend left = 0] (h);
		\end{tikzpicture}
	\end{center}
\end{figure}

\begin{enumerate}[label=(\roman*)]
	\item Construct the adjacency matrix of this graph. {\bf Tip:} I included a snippet of code you can use to create a matrix in \LaTeX.
	
	\solution
	\vspace{-15pt}
	\[
	\begin{bmatrix}
	? & ? \\
	? & ?
	\end{bmatrix}
	\]
	
	\item Construct the adjacency list of this graph.
	
	\solution
	
	\item BFS this graph starting from the node $1$.  Always choose the lowest-numbered node next.  Draw the BFS tree and label each node with its distance from $1$.
	
	\solution

\end{enumerate}

\end{enumerate}
	
\newpage
\begin{prob}DFS and Topological Ordering\end{prob}

\begin{figure}[h!]
	\begin{center}
		\begin{tikzpicture}[>= latex, node distance = 1.5 cm and 2.5cm,on grid, thick,state/.style ={circle,top color =white,bottom color = gray!20,draw,black,text=black,minimum width =1.0 cm}]
		\node[state] (a) {a};
		\node[state] (b) [right =of a] {b};
		\node[state] (c) [right =of b] {c};
		\node[state] (d) [below =of a,] {d};
		\node[state] (e) [right =of d] {e};
		\node[state] (f) [right =of e] {f};
		\node[state] (g) [below =of d, yshift=-10] {g};
		\node[state] (h) [right =of g] {h};
		\node[state] (i) [right =of h] {i};
		\tikzset{every node/.style={fill=white}} 
		\path[->] 	(a) edge  				(b)
		(a) edge				(e)
		(b) edge				(c)
		(c) edge [bend right=35]	(a)
		(d) edge				(a)
		(d) edge				(e)
		(e) edge				(b)
		(e) edge				(c)
		(e) edge				(f)
		(e) edge				(g)
		(f) edge				(c)
		(g) edge				(d)
		(g) edge				(h)
		(h) edge				(e)
		(h) edge				(i)
		(i) edge				(f);
		\end{tikzpicture}
	\end{center}
\end{figure}

Consider running depth-first search on this graph starting from node $a$.  When there are multiple choices for the next node to visit, go in alphabetical order.

\begin{enumerate}[label=(\alph*)]
	\item Label every edge as either tree, forward, backward, or cross.
	
	\solution
	
	\item Give the post-order numbers of all vertices
	
	\solution
	\begin{table}[h!]
		\centering
		\begin{tabular}{|l|l|l|l|l|l|l|l|l|}
			\hline
			a & b & c & d & e & f & g & h & i \\ \hline
			&  &  &  &  &  & &  &   \\ \hline
		\end{tabular}
	\end{table}
	
	\item Is this graph a DAG?  If so, give a topological ordering.
	
	\solution
	
\end{enumerate}

\vfill
\newpage
\begin{prob}Graph Properties\end{prob}

\renewcommand{\deg}{\mathit{deg}}

Consider an undirected graph $G=(V,E)$. The {\em degree} of a vertex $v$ is the number of edges adjacent to $v$---that is, the number of edges of the form $(v,u) \in E$.  Recall the standard notational convention that $n = |V|$ and $m = |E|$.

\begin{enumerate}[label=(\alph*)]
	\item Prove by induction that the sum of the degrees of the vertices is equal to $2m$.
	
	\solution
	
	\item Prove that there are an even number of vertices whose degree is odd.
	
	\solution
	
	\item Let $v \in V$ be some vertex whose degree is odd.  Prove that there exists another vertex $u \in V$ such that $u$ has odd degree and there is a path connecting $v$ and $u$.
	
	\solution
	
\end{enumerate}



\end{document}
